\documentclass[parskip=full]{scrartcl}
\usepackage[utf8]{inputenc} % use utf8 file encoding for TeX sources
\usepackage[T1]{fontenc}    % avoid garbled Unicode text in pdf
\usepackage[english]{babel}  % english hyphenation, quotes, etc
\usepackage[colorlinks=true,linkcolor=blue]{hyperref}       % detailed hyperlink/pdf configuration
\usepackage{graphicx}       % provides commands for including figures
\usepackage{csquotes}       % provides \enquote{} macro for "quotes"
\usepackage{enumitem}
\usepackage{multicol}
\setlength{\columnsep}{4cm}
%\usepackage{lscape}	% provides landscape portrait

\usepackage{pdflscape}	% provides horizental landscape portrait


\usepackage{pdfpages}	% add another pdf in  Latex

\usepackage{tikz}

\usepackage{float}


\usepackage{verbatim}	% provides multi-line comments

\usepackage{afterpage}
\usepackage{ragged2e}
\usepackage[export]{adjustbox}
\newcommand\tab[1][1cm]{\hspace*{#1}}

\graphicspath{{images/}}


\title{\Huge \textbf{HePICS Implementation Document}}
\date{\today \vspace{+10ex}}
\author{Andres Stober \\
	\and Mehyar Cherni \\
	\and Ibrahim Bouriga \\ 
	\and Linjuan Fan \\
	\and Bahaa Mahjane \\ }

\begin{document}

\maketitle
\thispagestyle{empty}

\begin{tikzpicture}[remember picture, overlay]
  \node [anchor=north west, inner sep=0.5pt, yshift=-20pt,xshift=20pt]  at (current page.north west)
     {\includegraphics[height=1.9cm]{Logo_KIT}};
\end{tikzpicture}

\begin{figure}[b]
\centering
\includegraphics[width=0.45\textwidth, center]{boardimage}
\end{figure}

\pagebreak

\tableofcontents
\thispagestyle{empty}
\pagebreak



\section {Introduction}
	\tab In this document you will find a full description of the work flow in the implementation phase as well as the deviations and changes made on the class diagram. Those changes are sometimes necessary and have convincing reasons that will be mentioned in details.
	\subsection{Workflow}
	\tab In order to distribute tasks and maintain good communication between team members, we use a scrum framework online 	\url{https://app.vivifyscrum.com} and a github repository \url{https://github.com/Turon42/hepics}.
	
	\subsection{Tasks}
	\tab The project can be devided in six parts:
	\tab \begin{itemize}
	\item GUI \ref{GUI}
	\item Image and Neural Network Model  \ref{Image and Neural Network Model}
	\item Scheduler \ref{Scheduler}
	\item Connecting PFGA \ref{Connecting FPGA}
	\item Operation Modes  \ref{Operation Modes}
	\end{itemize}
	
\pagebreak
	
\section{GUI} \label{GUI}
	@Linjuan Fan 
	\subsection{Window}
	\subsubsection{HepicsGUI}
	    \tab The class HepicsGui has been renamed as class \textbf{Main} which still contains the main method and represents the entry point to start the software by opening the welcome window.


    \subsubsection{WelcomeWindow}
        \tab The Class WelcomeWindow remains its functionalities: displaying information about the software, showing a welcome picture and as a start entry for main window. 
        Its attributes have been changed into two labels (for information text and welcome image), a push-button (for open the main window) and the pointer of main window. 
        The constructor is remained and a slot method named “openMainWindow” is added.
        
        \begin{figure}[H]
	        \centering
	        \includegraphics[scale=0.55]{images/welcomewindow.PNG}
	        \caption{Welcome Window}
	        \label{fig:my_label}
	    \end{figure}
        
    \pagebreak
    
    \subsubsection{MainWindow}
        \tab The class MainWindow specifies its attribute \textbf{section} as three pointer of control section, platform section and image section.
        
         \begin{figure}[H]
	        \centering
	        \includegraphics[scale=0.7]{images/main_window.PNG}
	        \caption{Main Window}
	        \label{fig:my_label}
	    \end{figure}       
        
    \subsection{Observer Pattern}
        \tab The class Observer, ObserverManager, ResultObserver, StateObserver have been removed. Instead of having observerable objects and observers, and registering them, Qt provides two high level concepts: \textbf{signals} and \textbf{slots}.  
        \begin{itemize}
            \item A signal is a message that an object can send, most of the time to inform of a status change.
            \item A slot is a function that is used to accept and respond to a signal.
        \end{itemize}
    
    \subsection{Section}
    \subsubsection{Section}
        \tab This class has been removed, because all the three sections are inherited from \textbf{QWidge}, which has been defined in Qt Gui library.
        
    \subsubsection{ControlSection}
        \tab This class adds the pointers for a progress bar, a label for result and three push-buttons as attributes. The \textbf{setLabel} method is now unnecessary, because the setting of the label and other widgets is implemented in its constructor. Moreover, the methods \textbf{handleButtonhandleButton} is extended as three different slots methods for each buttons.
        
         \begin{figure}[H]
	        \centering
	        \includegraphics[scale=0.7]{images/control_section2.PNG}
	        \caption{Control Section}
	        \label{fig:my_label}
	    \end{figure}       
        
        \subsubsection{Platform\_Mode\_Section}
        \tab This class adds two labels, three check-boxes and a combo-box as attributes. The \textbf{setLabel} method is also unnecessary, and the setting of all the widgets is implemented in its constructor. The slots methods \textbf{enableModeComboBox} is added for manage the sections between platforms and operation mode.
        
        \begin{figure}[H]
	        \centering
	        \includegraphics{images/platform_mode_section1.PNG}
	        \caption{With disabled combo box}
	        \label{fig:my_label}
	    \end{figure}
	    
	   \begin{figure}[H]
	        \centering
	        \includegraphics{images/platform_mode_section2.PNG}
	        \caption{With enabled combo box}
	        \label{fig:my_label}
	    \end{figure}
        
        \subsubsection{ImageSection}
        \label{sec: ImageSection}
        \tab This class adds a label for thumbnail, a combo-box, three buttons and a file dialog. An additional slot method \textbf{showThumbnail} is used for show the thumbnail of the current selected image. 
        
        \begin{figure}[H]
	        \centering
	        \includegraphics{images/imagesection1.PNG}
	        \caption{Image Section}
	        \label{fig:my_label}
	    \end{figure}
	    
	   \begin{figure}[H]
	        \centering
	        \includegraphics{images/imagesection2.PNG}
	        \caption{Images List}
	        \label{fig:my_label}
	    \end{figure}
	    
	    
        
        \subsubsection{FileLoader}
        \tab This class is now unnecessary, because it is replaced by \textbf{QFileDialog} which has been defined in Qt library and it is contained in ImageSection\ref{sec: ImageSection} as an attribute. 




\pagebreak

\section {Image and Neural Network Model} \label{Image and Neural Network Model}
	@Ibrahim Bouriga @Mehyar Cherni
	
	\subsection {Image Model}
	\tab The Image Class represents the \textbf{input} as well as the \textbf{filter} used in the different layers.\\
	The Image Manager Class however was removed in the implementation because all of its methods can be fully implemented as static ones in the Image Class.\\ As a result the use of these methods does not require any new instance object to be created and thus makes less use of memory. \\ Classes InputImage and ResultImage were not needed as results are only to display as text for the moment, as for input images, they are simply represented as instances of the class Image.\\ Here is a list of static methods implemented in Image:
	\begin {itemize}
		\item load\_image\_color: load an image
		\item load\_image: load an image an resize it if necessary
		\item load\_image\_stb: uses an external library to load an image. See \ref{External Libraries}
	\end{itemize}
	\subsection {Image and Result Saving}
	\tab The class DataSaver is meant to work as a data base, it consists of one map whose keys are images' unique identifications id, and whose values are results, which are represented as well as an object on their own. DataSaver allows the program to store input images and their results, retrieve them, delete them, or change their values. It also offers the possibility of saving the results of one classification in a text file. \\The aggregate function stores all classnames and percentages in a new map named global, converts the map into a vector to sort it, then chooses the 4 biggest values and returns them as a Result object.
\\The class Result wasn't included in the design, and was added for the purpose of making the storage clearer and more flexible. It is meant to represent the results of a classification of a single input image as follows. Each instance of the object Result contains a map, that stores classnames represented as strings, and links them to their values stored as floats. Functionalities included are saving a classname and its percentage into the map. To control the number of classnames needed, we added the int i who counts the number of elements which are currently saved. This class offers the possibility to write results as plain text to display it on the GUI as such, classname1 : percentage1, classname2 : percentage2 etc ... 
	\subsubsection {Assistant}
	\tab Class ClassificationAssistant has been renamed into Assistant. It serves the purpose of providing and storing all data given by the user through the GUI to the system. The weight file is loaded directly into the network through a configuration file, so it doesn't belong to the assistant anymore. The boolean function isclassified() has been removed since it hasn't been needed. The classnames file is set in the assistant, which will then be able to load them into a list of strings. Input images are stored into a list as well, with the possibility of deleting them by selection, or clearing the whole list.

	\subsection {Network Model}
	\tab The Network Model consists of the network and the layers:
	
		\subsubsection {Network}
		\tab The Network Class was implemented as it was introduced in the class diagram. However displaying the topology  as designed was not part of the implementation due to the fact that only the alexnet architecture will be deployed in the classification process. To display the topology of the neural network, an image whose path has been set from the beginning is displayed, in order to give an idea about the structure of the implemented neural network.
		The architecure (See \ref{fig:alexnet-architecture}) itself was described in a comprehensible configuration file, which will be parsed (See Parser \ref{Parser}) to extract relevant information about the network.
		
		\subsubsection {Layer}
		\tab The Layer Class stores information such as the number of inputs and outputs as well as the layer type and define the forward function which is implemented by derived classes:\\
		\begin{itemize}
			\item Convolutional Layer
			\item Max Pool Layer
			\item Connected Layer
			\item Softmax Layer
		\end{itemize}
		\begin{figure}
			\centering
			\includegraphics[width=1.0\textwidth]{alexnet-architecture}
			\caption{alexnet architecture}
			\label{fig:alexnet-architecture}
		\end{figure}
		
		\subsubsection {Activation}
		\tab The Activation Class implements the following activation functions:
		\begin{itemize}
			\item ReLu $g(z) = max{0, z}$
			\item Linear $g(z) =z$
			\item Tanh  $g(z) = (e^z -e^{-z}) / (e^z + e^{-z})$
			\item Sigmoid $g(z) = 1 / (1 + e^{-z})$
		\end{itemize}
		
	\subsection {New Classes}
		\paragraph{a.Parser} \label{Parser}
		\tab This class was created to parse and extract information related to the network from the alexnet architecture (See \ref{fig:alexnet-architecture}).
		\begin{itemize}
			\item parse\_convolutional: creates a convolutional layer from the convolution section in the configuration file
			\item parse\_maxpool: creates a maxpool layer from the convolution section in the configuration file
			\item parse\_connected: creates a connected layer from the convolution section in the configuration file
			\item parse\_softmax: creates a softmax layer from the convolution section in the configuration file
			\item load\_network: creates the network and initialize layers
		\end{itemize}
		
		\paragraph{b.Utils}
		\tab This class implements static methods that does logically not belong to any object. For instance file and memory errors. 
		
		\paragraph{c.Options\_list}
		\tab This class reads data from the configuration file line by line.\\
		It models a hash set with key words and associated values. The extracted values will be stored in lists (See \ref{Structures})
		
	\subsection {External Libraries} \label{External Libraries}
		\begin{itemize}
			\item stb\_image
			\item GEMM
			\item BLAS
			\item IM2COL
		\end{itemize}
		
	\subsection{Structures} \label{Structures}
		\tab Some availabe structures were implemented again in order to avoid unexpected behaviour like:
		\begin{itemize}
			\item List
			\item Node
		\end{itemize} 
		
\pagebreak

\section{Scheduler} \label{Scheduler}
	@Mehyar Cherni\\
	\tab Class Scheduler has been set for the service of enabling and selecting the platforms corresponding to the selected mode. Following this logic, an enumeration type has been defined under the name Platform.
	\begin {itemize}
		\item cpu, 
		\item gpu, 
		\item fpga
	\end{itemize}
	The class also consists of two other vectors of booleans. The first one, named platforms, represents the platforms that the user has chosen to enable. The second, named use\_platforms, defines which platforms are to run the code on, but only after the user chooses the operating mode. The choice of the use of platforms follows a comparison of the three platforms on three different levels, supposing that the code is meant to run parallel and contains mainly vector operations.
	\begin {itemize}
		\item Power Consumption : cpu>gpu>fpga
		\item Performance : fpga>gpu>cpu
		\item Energy Efficiency : fpga>cpu>gpu
	\end{itemize}
Next to accessing information about the mode and currently enabled/used platforms, Scheduler also contains a vector of workers, whose size has to be defined while creating an instance of the class. Each worker can be assigned to execute a task on a selected platform, which he would only do if, this platform is allowed to use after the choosing process. \\A group of methods has been also made available for a possibly future parallel implementation of the classification
	\begin {itemize}
		\item bool isWaiting() : gives the status of the worker provider
		\item void next() : provides the next worker, when not waiting
		\item void wait() : sets the provider into wait mode
	\end{itemize}
As for the progress bar, it seems that its implementation is more complicated than expected, which delayed its implementation to a future time due to functionalities with higher priority.
	
	
\pagebreak

\section{Connecting FPGA} \label{Connecting FPGA}
@Andres Stober

Connecting to the FPGA proved more difficult than expected. Due to lack of experience with the platform we were unware of the soc board holding another linux that we had to work with. To use the FPGA we had to write a service to which we can connect via ethernet and which uses the FPGA. Therefore a Socket class was added. High complexity led to multiple OpenCL classes. Also padding was an issue that had to be addressed to reduce branching in the FPGA

\subsection{Socket}

Sockets provide the communication mechanism between two computers using TCP. In our project we created a socket because we need the communication with the PC that is connected with the FPGA. 

In the socket.h class we used these methods:

\begin {itemize}
	\item void set\_tcp\_nodelay
	\item void receive
	\item void send
\end{itemize}

set\_tcp\_nodelay method: this method is intended to disable/enable segment buffering so data can be sent out to peer as quickly as possible, so this is typically used to improve network utilisation

send method:  is used only when the socket is in a connected state. This method codes the input image parameters into a byte Array using vector<uint8\_t> to send it to the FPGA.

receive method: decoding the byte array(vector<uint8\_t>).

\pagebreak

\subsection{OpenCL}

OpenCL (Open Computing Language) is the open, royalty-free standard for cross-platform, parallel programming of diverse processors found in personal computers, servers, mobile devices and embedded platforms. In our project we use this framework for programming heterogeneous systems that contain a combination of CPU, GPU and FPGA.

In “OpenCL.h” file we used Resource Acquisition Is Initialization or RAII technique, therefore we have 6 classes in this “OpenCL.h” file. 
RAII technique binds the life cycle of a resource that must be acquired before use ( open socket…) to the lifetime of an object. 
With this technique we want to guarantee that the resource is available to any function that may access the object. It also guarantees that all resources are released when the lifetime of their controlling object ends, in reverse order of acquisition.
 
\pagebreak

\subsection{Zero-Padding-Concept}

A convolution neural network is a sequence of layers. We used 3 types of layers for the implementation of the convolution neural network:

\begin {itemize}
	\item Convolution Layer
	\item MaxPooling-Layer
	\item Fully-Connected-Layer
\end{itemize}

Zero-padding refers to the process of symmetrically adding zeroes to the input matrix. 
It's a commonly used modification that allows the size of the input to be adjusted to our requirement. We used it in designing the CNN layers when the dimensions of the input volume needs to be preserved in the output volume.

For the Zero-padding Method we used these 3 methods:
\begin {itemize}
	\item INPUT [WidthxheighxRGB(=3)] will hold the raw pixel values of the image.
	\item CONV This method may result in volume such as [widthx3heighx(number of filter * 3(RGB))].
	\item RELU returns max(0, x). Size of the volume.
\end {itemize}
    
About the Matrix Implementation:
Accepts a volume of size W1 * H1 * D1
Requires four hyperparameters: 
Number of filters K
their spatial extent F
the stride S
the amount of zero padding P
Produces a volume of size W2 * X2 * D2, where:
W2 = (H1 - F + 2P) / S + 1
D2 = K

\pagebreak


\section{Operation Modes} \label{Operation Modes}
	@Mehyar Cherni\\
	\tab Classes Mode, and the three different modes that inherit from it were set to be represented as a public enumeration type inside the class Scheduler as such \\ Mode
	\begin {itemize}
		\item standard,(initial state, before selecting the mode)
		\item high\_performance 
		\item low\_power
		\item energy\_efficient
	\end{itemize} 
	 , since they contain no functionality, but only describe states.


\pagebreak


\end{document}
